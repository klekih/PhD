\documentclass[a4paper,12pt,twoside]{book}
\usepackage[english]{babel}
\usepackage{blindtext}
\usepackage[utf8]{inputenc}
\usepackage{amsmath}
\usepackage{apacite}
\usepackage{natbib}
\usepackage{listings}

\bibliographystyle{apacite}
\pagestyle{myheadings}

\title{Analyzing opportunities arround city traffic optimization }
\author{Toma Becea}

\begin{document}

\maketitle

\clearpage

\section{Introduction}

The traffic in a city is a very complex phenomenon. The parties involved are of many types: cars, pedestrians, buses, trams, taxi cabs (they are arugably cars, too but there might be small differences, e.g. they are allowed to drive on public transport lanes), bicycles, e-scooters (again, they are arguably same as bicycles), etc. Then there are rules on how they move: traffic lights for cars in junctions, across roads for pedestrian crossings, different traffic lights for buses and trams, etc. Within this rule-bounded context is the free will of every intelligent decision maker (it might be a human or it might be a self driving car with algorithms which we can deem as being intelligent from the perspective of behaving and moving within the traffic), the individual randomness unified into a collective orderness.

But the devil is in the details: the orderness we see when looking across the entire city uncovers many opportunities. The architecture of many old cities is not meant for current traffic needs. The behavior of residents and dwellers of a city is not optimized for collective good but for personal comfort. The heterogenous and continous development of a city might add pain to the picture. Highly-accelerated developing areas or neighborhoods pose problems which city administrations are slow to understand and tackle with. Those, so far, are problems on the macro scale of a spectrum. On the other side of the spectrum, called micro, we can find another set of opportunities which can be modeled and analyzed through computer based simulations. Junction design, traffic lights programming and synchronization, pedestrian crossings and their influence on the surrounding driving lanes, bus stops and their influence, side walks design, lanes formation and one way drives, etc. Those are localized problems, focused on small geographical areas.

The current chapter aim to find and define technical means on exploring intricacies within the traffic of a city. Software simulation is a broad term which encompasses any idea or way of using a computer to mimic as closely as possible a real phenomenon, with or without the help of a mathematical apparatus. This entire chapter is centered around various tehniques of using a computer for traffic simulation. There are a number of existing solutions to simulating traffic using software abilites, on various places on the macro-micro spectrum, as well as on other spectrums like discrete event vs continous simulation. Those will be explored and briefly categorized.

\section{Abstract}

The current chapter is aiming to propose a novel solution for simulating traffic in a city and exploring the opportunities which this solution makes possible to address. To describe it in few words, before going in full details on it during section \ref{sec:microservicesbasedapproachtrafficsim}, it is a discretes event based simulation, using a distributed microservices architecture where actors (defined as independent and intelligent participants in a traffic) are moving across a city and interacting with one another. They are a number of different instances running in isolation as containers. Their interaction is made via a single instance, called city simulator, which keep various statistics about the actors which are live and moving. On top of those there are few other supporting utilities, not essential for the current subject but mandatory for the simulation to run. They are a map and routing service which offers a route between two points of a city to those actors which need it and a service which renders a web page with the map of the city where the simulation takes place and the gradual occupation of its streets.

The second area of this chapter is to analyze opportunities for optimizing the traffic within a city using the aforementioned simulation solution. Those ideas will revolve around the core concepts which form the foundation of the microservices based approach for traffic simulation: discrete and independent actors (also randomness is included into their behavior), decoupled in their logic and choice, interacting via discrete events based on their selfish interest (their behiavor which this simulation is modeled for is their interest in navigating their route as fast as they can, using the shortest path between two points, departure and arrival ones).

Following concepts will be discussed in order to explore the opportunities for optimizing the traffic in a city. They are to be found in section \ref{sec:citytrafficoptimizations} and a brief list of them is presented here:

\begin{itemize}
    \item Modeling various actors with very different behaviors. For example a pedestrian and a city bus. Or a bicycle and a garbage truck.
    \item Modeling various relations between different actors. An example of such a relation can be contained actors: a pedestrian can take a tram.
    \item Introducing exception actors and their influence in the surrounding traffic. Firefighters truck moving throughout the city is one example. Cars stopped on the road is another.
    \item Localize hot areas or bottlenecks of a city: junctions, streets, bridges.
    \item Computation of various numbers and statistics. Few examples: average speed on street intervals, cars density per area and/or time.
    \item Introduce geographical differences in simulation and study their influence: a roundabout instead of a junction, a new lane on a street.
\end{itemize}

\section{Current state of traffic simulation}

\section{A microservices based approach for traffic simulation}
\label{sec:microservicesbasedapproachtrafficsim}

\subsection{Prerequisites}
\label{subsec:prerequisites}
In order to discuss how a discrete-based, microservices architectured simulation works, few prerequisites are needed to set the base.

First definition we need is for setting the fact that we are using geographical coordinate system. It doesn't matter which one, what it matters is that all actors are using the same set of geographical coordinates and same projection (or to know if there should be a conversion). For example, they could be EPSG:4326 or EPSG:3857. To define this in an appropiate way, equation \ref{eq:coordinates} will be used. Its loosness allow any implementation to treat coordinates using a built in language data type (e.g. \textit{double} in C\#).

\begin{equation}
\label{eq:coordinates}
    C = \{ c | c \in \Re, c \geq 0 \}
\end{equation}

Simulation wise, actors do not have any \textit{intent}. They are just \textit{dumb} programs which need to be programmed to have any small \textit{intention}. To give them a \textit{sense} of selfishness they are programmed to navigate between two random points. Therefore points need to be defined as a pair of coordinates. The implementation of this will have to take care of their order or of knowing which one represents latitude and which one longitude. Equation \ref{eq:coordinatespair} defines such a pair of coordinates. Furthermore we define a way which introduce randomness into an actor's behavior: the random choice of two points. The actor needs to have two points predefined, $Pd_1$ and $Pd_2$. Those are treated as an aproximative rectangular area, out of which other two points will be randomly chosen: $Pr_1$ and $Pr_2$. Equation \ref{eq:funcrandom} defines this function, without entering into the details of how such a random function would work.

\begin{equation}
\label{eq:coordinatespair}
    P = \{ (p1, p2) | p1 \in C, p2 \in C \}
\end{equation}

\begin{equation}
\label{eq:funcrandom}
\begin{split}
    & f_{rand}: P \rightarrow P, \\
    & f_{rand}(Pd_1, Pd_2) = rand (Pd_1, Pd_2) \Rightarrow (Pr_1, Pr_2)
\end{split}
\end{equation}

So far an actor has two random points and now it needs a route. For this the final function we need to define as a prerquisite is in equation \ref{eq:funcroute}. Essentially this function takes two points, in this case the two random points obtained earlier, in equation \ref{eq:funcrandom}: $Pr_1$ and $Pr_2$ and outputs a series or a list of points where first point is the first random point, $Pr_1$ and last point is the second random point, $Pr_2$. Between them a number of points represents the path which the actor should follow in order to travel from its first (randomly choosen) point to its second (randomly choosen) point. Translated into a more intuitive description, those points are the instruction which any maps application outputs when asked for a route. For example OpenStreetMap, \cite{openstreetmap} is a map provider and Graphhopper, \cite{graphhopper} is a route provider which uses OpenStreetMap as the default map.

\begin{equation}
\label{eq:funcroute}
\begin{split}
    & f_{route}: P \rightarrow [P], \\
    & f_{route}(Pr_1, Pr_2) = route (Pr_1, Pr_2) \Rightarrow [Pr_1, Pc_1, Pc_2, ..., Pc_n, Pr_2]
\end{split}
\end{equation}

\subsection{Microservices - city actor}

Having defined the prerequisites from section \ref{subsec:prerequisites} we move on to present the first type of microservice from our architecture: the (city) actor. A city actor is meant to represent a specific behavior of an entity which moves across the city. To create such an actor it must be infused with specific behaviors based on the reality it wants to mimic. The prevalent example of this chapter will be a car. Therefore the prerequisites from section \ref{subsec:prerequisites} will suit a car's purpose to obtain a route using a predefined area where the car is allowed to wander. To further isolate its movements and decisions an algorithm is supposed to be used at its core. Listing \ref{lst:caractoralgorithm} presents it.

\begin{lstlisting}[caption=Basic algorithm for a car actor, label=lst:caractoralgorithm]
\\ Get two random points
(pr1, pr2) = rand (pd1, pd2)
\\ Get the route
R = route (pr1, pr2)
\\ Set the position to the first point
P my_position = pr1
int index = 0
while (index < R.length) {
    int speed_coeff = get_info(R[index])
    my_position = advance(speed_coeff, R, index)
    report(my_position)
}
\\ Report that the route has been finished
report(no_position)
\end{lstlisting}

Few remarks about the core algorithm showed in listing \ref{lst:caractoralgorithm}. There are few pieces which are missing from the prerequisites but are discussed into section \ref{subsec:citysimulator} because they are related to the communications between a car actor and the city simulator. Those pieces are the calls to following methods: \textit{get\_info} and \textit{report}. The third one, \textit{advance} is the place where each actor is going to be personalized with own rules and behavior. Those aspects will be discussed into the subsection \ref{subsec:actordiffbehaviors}.

The core algorithm needs to be run on continuosly basis, with a standard frequency (e.g. 500 ms). Varying this number, also called simulation step can be an option for speeding or slowing the simulation running time. The \textit{advance} method is supposed to compute where to go next and it is responsible for incorporating the \textit{speed\_coeff} coefficient obtained from the city simulator. This coefficient is meant to express the density of cars on the street (if the actor which needs it is a car. In the case where it is a pedestrian the coefficient will be different. This way it is intuitively described the fact that a pedestrian walking on the sidewalk has nothing to do with the cars which are driving on near lanes and viceversa, a car driving on some lane doesn't care how many pedestrians are on the sidewalk near it) and it varies from expressing a totally free road to a road blocked by the numbers of cars waiting to pass through a junction.

A written, more expansive description of this algorithm is as follows. For easier following of the simulation and for the current paper all actors are bounded to a certain area, expressed as two defined points $Pd_1$ and $Pd_2$. Those points are treated as the opposing corners of a rectangle and another two random points are chosen such that they are wihtin the rectangle: $Pr_1$ and $Pr_2$. The actor will then proceed to obtain a route, from another service, in order to know how to travel from $Pr_1$ to $Pr_2$. The route obtained is a series of another points, $Pc_1$ tp $Pc_n$, starting from $Pr_1$, called departure point, right to the finish point: $Pr_2$. They can be organized into subsets. A subset is a number of consecutive points which are treated as one instruction. Those instructions are meant to offer an easier guide, vocal or visual, for a human traveler. However, there are no human travelers into the simulation so the instructions have found another useful role. Usually an instruction is a limited length of a street during which the traveler doesn't need to do any change of direction. As soon as a junction must be traveled, there will be another instruction. Therefore this subset of the entire vector of points will become the interval which the actor reports as being traveling on to the city simulator. The city simulator, having such reports from all actors, will respond with the a number called speed coefficient, which reflects how many other actors are on the same interval and it will influence the traveling speed. The $advance$ method is meant to compute how the actor will advance and is the main place, though not the only one, where the behavior of different types of actors will be implemented. This method also takes care of handling the speed coefficient. Once the method computes the next position where the actor will be, that position will be reported back to the city simulator. This logic repeats itself as long as the actor has points and instructions which haven't been traveled yet. At last, the actor will report that it has been finished its route so that the city simulator will take into account one less actor.

\subsection{Microservices - city simulator}
\label{subsec:citysimulator}

In this first chapter the city simulator is a separate entity which takes care of all the reports the actors send to it and also respond with every query about a certain route. Thus, it will need a way to keep this data. There are few assumptions around this which need to be taken into consideration:

\begin{itemize}
    \item The set of points $P$ represents routes whose interpretation are portions of streets;
    \item A report sent by an actor contains a set of points and an indication if the actor is currently travelling on that route or if it went off it;
    \item A set of points which was received from an actor can be a subset of another set of points received earlier from another actor;
    \item A set of points which was received from an actor can be a superset of another set of points received earlier from another actor;
    \item An actor can ask various information for any set of points it needs;
    \item Any information which will be conveyed to an actor must take into account any subset and superset which might already exist;
    \item An actor will send in each report an piece of information representing its type (car, person, etc.);
\end{itemize}

To illustrate above points the speed coefficient will be discussed. The first and uttermost information which an actor needs is representing the concentration of other actors of its type which around it or on a certain piece of road. For example, a car needs to know how many cars are around it, to know how fast it can advance. Or it might need to know how many cars are on a certain piece of road to asses if it can find a faster route. While the latter example is an intuitive one the former needs a little more explanation. Being a simulation, an actor doesn't know how many other actors are around it. In a real scenario the driver just have to look around but here there are no drivers and no looking around. One solution is for an actor to ask the city simulator how many other actors of its type are on the same street it's currently traveling on. Therefore, this information will need to be baked in the each actor type. For example a car will need to adjust its nominal speed with the coefficient which it received from the city simulator.

\subsection{Implementation}



\section{City traffic optimizations}
\label{sec:citytrafficoptimizations}

\subsection{Actor with different behaviors}
\label{subsec:actordiffbehaviors}

\bibliography{wiki}

\end{document}
