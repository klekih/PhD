\documentclass[conference]{IEEEtran}
% The preceding line is only needed to identify funding in the first footnote. If that is unneeded, please comment it out.
\usepackage{url,apacite}
\usepackage{amsmath,amssymb,amsfonts}
\usepackage{algorithmic}
\usepackage{graphicx}
\usepackage{textcomp}
\usepackage{xcolor}
\bibliographystyle{apacite}
\def\BibTeX{{\rm B\kern-.05em{\sc i\kern-.025em b}\kern-.08em
    T\kern-.1667em\lower.7ex\hbox{E}\kern-.125emX}}
\begin{document}

\title{A microservices based approach for city traffic simulation}

\author{\IEEEauthorblockN{1\textsuperscript{st} Given Name Surname}
\IEEEauthorblockA{\textit{dept. name of organization (of Aff.)} \\
\textit{name of organization (of Aff.)}\\
City, Country \\
email address}
\and
\IEEEauthorblockN{2\textsuperscript{nd} Given Name Surname}
\IEEEauthorblockA{\textit{dept. name of organization (of Aff.)} \\
\textit{name of organization (of Aff.)}\\
City, Country \\
email address}
}

\maketitle

\begin{abstract}
Simulating traffic in a city is a difficult task. Current paper is proposing a novel way using state of the art in distributed systems: microservices orchestration. The design is centered around two types of actors: a city simulation actor (a single instance of it) which keeps track of occupied streets and and their gradual occupation. The other actor type (multiple instances) is the car which travels across the city.
\end{abstract}

\section{Introduction}
A solution which simulates car, pedestrian, etc. traffic in a given city may reap many benefits. It can help understand patterns of traffic and its flow. It can help unerstand the particularities and pecularities of a city's streets arrangements. It can help identify bottlenecks. It can help find solutions to rush problems and explore them. But for those areas to be tackled, appropiate methods of simulating traffic must be found. We are proposing a novel way of simulating traffic based on the microservices orchestration concept.

We define a \textbf{(city) actor} as being a independent entity which chose to move between two geographical points within a city. It can be a car, a pedestrian or a bike. We also define the \textbf{city simulator} as being a single entity (subject to distributed and load balacing services) which keeps data about the city (e.g. streets with city actors on them).

\section{Existing solutions and ideas}
(Summary of other papers with traffic simulation)

\section{Microservices}

\subsection{Current state}
Microservices are not a new concept. The idea behind them has existed since Linux kernel has started to be enriched with a concept called namespaces \cite{wiki:linuxns}. This allows one set of processes to see one set of resources while another set of processes see another set of resources, where resources might be, but not limited to, process IDs, file names and network resources. Those linux kernel abilities form the base of containers.

Thus, a container is a small set of processes which run in isolation. They allow packaging a linux distro, a set of libraries and skd and on top of those custom code. This forms an image, which is essentially a tar gzipped file. Once the build process of an image is finished, it can be spinned up in one or more running containers. The custom code written and embedded in the image is running in parallel in each container.

To go to solution for building, manipulating and running images is Docker \cite{docker}. It allows easy software installation, it works cross platform and it offers a smooth experience most of the time.

\section{Implementation}

The entities which are participating into a traffic simulation are called actors. They are two: the city actor and the city simulator. The supporting containers are not themselves part of traffic simulation but they do have an important and supporting role.

\subsection{City simulator}
The most common and easy solution to share data across all the city actors is to have a centralized store to keep it. City simulator acts as a centralized store for all other city actors. In the current implementation the city simulator keeps a set of data which can be described as a list of pairs, each pair having a line and a real number, called density.

The density is defined as the number of actors (cars) which are at a given moment present on a given segment of street. If the city simulator has no entry of a street segment then it will consider the density as being 0, i.e. there is no actor on that street. The density is modeled as a unsigned integer.


(Future enhancements: integration into car actor for distributed consensus and data exchange)

\subsection{Car actor}
The city actor represents a moving actor within a city. It can be a car which moves across the city, it can be a bike or it can be a pedestrian. Its naming suggests that it can be any entity or living being which moves within a city and interacts with the other entities or affects the other entities in some manner. A pedestrian would directly interact with other pedestrians but not with cars. A pedestrian would indirectly affect other cars by willing to 
(Routing logic)

\subsection{Interactions}

\section{Orchestration}

\subsection{Current state}

\subsection{Implementation}

\subsection{Docker compose}

\subsection{Kubernetes}

\section{Design choices}

\subsection{Tools and programming languages}
\begin{itemize}
\item Go programming language
\item Docker
\item Docker compose
\item OpenSourceMap
\item Graphhopper
\item Web page choices (web page will become an actor)
\end{itemize}

\subsection{Choices and rationales}

\subsection{Alternatives}
(Actor model and consacrated frameworks)

\subsection{Networking design}

\subsection{Orchestration design}

\section{Results}

\subsection{Visual}

\subsection{Performance}

\section{Enhancements}
(Distributed database for multiple city simulators)

(More granular logic on actor behavior)

(More types of "junctions": classic junction for cars, pedestrian crossings, subways access stairs and elevators, bus stations, etc.)

(Contained actors: pedestrians in a bus, pedestrian in a subway)

\bibliography{wiki}
\vspace{12pt}

\end{document}
