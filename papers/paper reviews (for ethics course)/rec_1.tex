\documentclass{article}
% \IEEEoverridecommandlockouts
% The preceding line is only needed to identify funding in the first footnote. If that is unneeded, please comment it out.

\usepackage{geometry}
    \geometry{
        a4paper,
        total={170mm,257mm},
        left=20mm,
        top=10mm,
        }

\begin{document}

\title{Peer single blind review of "Web-Controlled Robot Based on Raspberry Pi" paper}

\author{}

\maketitle

\section{Plagiarism assesment}

The tool which was used: www.copyleaks.com. Results:

\begin{itemize}

    \item 23\% percent of content is similar to other sources
    \item 519 words copied
    \item less than 1 \% minor changes
    \item less than 1 \% of words with related meanings

\end{itemize}

\section{Abstract and introduction}

\begin{itemize}

    \item Provide a short description of the proposed improvement (solution) in the abstract so that subsequent sections are easier to understand within a context
    \item Expand the solution in the description instead of the lenghty Raspberry PI history
    
\end{itemize}

\section{Research specifics}

\begin{itemize}
    
    \item Add reference for "Getting Started" documentation of PiCamera Python library
    \item On Section II "Related work" describe the actual work done to connect the PiCamera to a Raspberry and how its Python library was used to interface with it
    \item Describe the novelty which the proposed solution brings besides interconneting various Raspberry compatible kits with a Raspberry computer.
    \item Provide actual data of how the research went during the control of the robot (communication round trip time, lag, time needed for IP registration, connection loss mitigation, bandwith benchmarks, resolution fallback on low bandwith, etc.)

\end{itemize}

\section{Results specifics}

\begin{itemize}

    \item Provide data interpretation of the tests run on the robot and its connectivity platform
    \item Provide and document the novelty of the proposed solution, together with its significance

\end{itemize}

\section{Recommended decision}

Fundamental changes are required for this paper, therefore the paper is rejected.

\end{document}
