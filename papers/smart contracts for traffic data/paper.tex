\documentclass[conference]{IEEEtran}
\IEEEoverridecommandlockouts
% The preceding line is only needed to identify funding in the first footnote. If that is unneeded, please comment it out.
\usepackage{cite}
\usepackage{amsmath,amssymb,amsfonts}
\usepackage{algorithmic}
\usepackage{graphicx}
\usepackage{textcomp}
\usepackage{xcolor}
\usepackage{hyperref}
\usepackage{balance}
\def\BibTeX{{\rm B\kern-.05em{\sc i\kern-.025em b}\kern-.08em
    T\kern-.1667em\lower.7ex\hbox{E}\kern-.125emX}}
\begin{document}

\title{Smart contracts for handling city traffic information}

\author{\IEEEauthorblockN{1\textsuperscript{st} Toma Becea}
    \IEEEauthorblockA{\textit{Automation and Computer Science Faculty} \\
    \textit{Technical University of Cluj-Napoca}\\
            Cluj-Napoca, Romania \\
            tomabecea@pm.me}
    \and
    \IEEEauthorblockN{2\textsuperscript{nd} Honoriu Valean}
    \IEEEauthorblockA{\textit{Automation and Computer Science Faculty} \\
    \textit{Technical University of Cluj-Napoca}\\
            Cluj-Napoca, Romania \\
            Honoriu.Valean@aut.utcluj.ro}
}

\maketitle

\begin{abstract}

    Blockchain technologies are becoming increasingly complex and thus are tackling more and more various use cases than decentralised finance which was the sole purpose of Bitcoin, first successful blockchain. Or they are tackling more and more use cases and thus their inherent complexity increases. Either way, one of the untouched places was IoT realm. Due to natural IoT constraints, like limited processing power and high volume of data blockchain technologies have stayed separately. However, the Ethereum 2.0 and IOTA are two distributed ledger technologies with two different approaches of handling a high volume of transactions without requiring specialized hardware. The current paper is aiming to explore the possibilites of those two technologies through the lens of a concrete use case: transfering city traffic information between traffic participants and reaching consensus towards optimizing the traffic. The data to be transferred has specific needs and formats. The traffic optimization consensus has to take into account various metrics and goals. Those two aspects need to be modeled on the exact blockchain technology via smart contracts.

\end{abstract}

\begin{IEEEkeywords}
    
\end{IEEEkeywords}

\section{Introduction}

\subsection{Terms}
\paragraph{Release}
\paragraph{Deploy}
\paragraph{Error rate}

\section{Current state}

\section{Solution}

\section{Conclusions}

\begin{thebibliography}{00}

    \bibitem{b1} Hall, Richard S., Dennis Heimbigner, and Alexander L. Wolf. "A cooperative approach to support software deployment using the software dock." Proceedings of the 21st international conference on Software engineering. 1999.

\end{thebibliography}

\end{document}
